\documentclass{article}
\usepackage[utf8]{inputenc}
\usepackage{../../shared/styles/conventions}
\usepackage{../../shared/styles/tutorial-style}

\tutorialtitle{Neural Networks}
\setheader{Neural Networks}

\begin{document}

\maketitle

\section{Summary from Slides}

\subsection{Multi-Layer Perceptrons (MLPs)}

\textbf{What is a Neural Network?}
\begin{itemize}
    \item Network of interconnected processing units (neurons)
    \item Each neuron computes a weighted sum of inputs plus bias
    \item Applies activation function to produce output
    \item Multiple layers create complex decision boundaries
\end{itemize}

\textbf{Mathematical Representation}:
For a single neuron: $z = \vw\tp\vx + b$, $a = \sigma(z)$

For multi-layer network:
$$\vz^{(l)} = \mW^{(l)}\vh^{(l-1)} + \vb^{(l)}, \quad \vh^{(l)} = \sigma(\vz^{(l)})$$

Where:
\begin{itemize}
    \item $\vh^{(l)}$: activations at layer $l$
    \item $\vz^{(l)}$: pre-activation values at layer $l$
    \item $\mW^{(l)}$: weight matrix connecting layer $l-1$ to $l$
    \item $\vb^{(l)}$: bias vector at layer $l$
    \item $\sigma$: activation function
\end{itemize}

\subsection{Common Activation Functions}

\textbf{Sigmoid}: $\sigma(z) = \frac{1}{1 + e^{-z}}$ (outputs between 0 and 1)

\textbf{ReLU}: $\sigma(z) = \max(0, z)$ (most common in hidden layers)

\textbf{Tanh}: $\sigma(z) = \frac{e^z - e^{-z}}{e^z + e^{-z}}$ (outputs between -1 and 1)

\textbf{Softmax}: $\sigma(z_i) = \frac{e^{z_i}}{\sum_{j=1}^K e^{z_j}}$ (for multi-class classification)

\subsection{Forward Propagation}

\textbf{Algorithm}:
\begin{enumerate}
    \item Start with input $\vh^{(0)} = \vx$
    \item For each layer $l = 1, 2, \ldots, L$:
    \begin{enumerate}
        \item Compute linear combination: $\vz^{(l)} = \mW^{(l)}\vh^{(l-1)} + \vb^{(l)}$
        \item Apply activation: $\vh^{(l)} = \sigma(\vz^{(l)})$
    \end{enumerate}
    \item Final output: $\yhat = \vh^{(L)}$
\end{enumerate}

\subsection{Automatic Differentiation (Backpropagation)}

\textbf{Purpose}: Efficiently compute gradients for parameter updates using chain rule

\textbf{Loss Function}: $\cL(\vtheta) = \frac{1}{n}\sum_{i=1}^n \ell(\yhati, y_i)$

\textbf{Chain Rule}: For composite function $f(g(x))$:
$$\frac{df}{dx} = \frac{df}{dg} \cdot \frac{dg}{dx}$$

\textbf{Backward Pass}:
\begin{enumerate}
    \item Compute output error: $\delta^{(L)} = \nabla_{\vh^{(L)}} \cL$
    \item For each layer $l = L-1, L-2, \ldots, 1$:
    \begin{enumerate}
        \item $\delta^{(l)} = (\mW^{(l+1)})\tp \delta^{(l+1)} \odot \sigma'(\vz^{(l)})$
    \end{enumerate}
    \item Compute gradients:
    \begin{align}
        \frac{\partial \cL}{\partial \mW^{(l)}} &= \delta^{(l)} (\vh^{(l-1)})\tp \\
        \frac{\partial \cL}{\partial \vb^{(l)}} &= \delta^{(l)}
    \end{align}
\end{enumerate}

\subsection{Training Process}

\textbf{Gradient Descent Update}:
$$\vtheta \leftarrow \vtheta - \alpha \nabla_{\vtheta} \cL(\vtheta)$$

\textbf{Key Challenges}:
\begin{itemize}
    \item \textbf{Vanishing Gradients}: Gradients become very small in deep networks
    \item \textbf{Exploding Gradients}: Gradients become very large
    \item \textbf{Local Minima}: Optimization may get stuck
    \item \textbf{Overfitting}: Model memorizes training data
\end{itemize}

\section{Practice Problems}

\begin{problembox}[title=Basic Neural Network Architecture]

Design a neural network for binary classification with:
\begin{itemize}
    \item Input: 4 features
    \item Hidden layer: 3 neurons with ReLU activation
    \item Output: 1 neuron with sigmoid activation
\end{itemize}

a) Write the mathematical equations for forward propagation
b) How many parameters (weights + biases) does this network have?
c) What is the output range for this network?
\end{problembox}

\begin{problembox}[title= Activation Function Analysis]

For the functions below, compute their derivatives:
a) $\sigma(z) = \frac{1}{1 + e^{-z}}$ (sigmoid)
b) $\sigma(z) = \max(0, z)$ (ReLU)
c) $\sigma(z) = \tanh(z)$

Explain why ReLU helps with the vanishing gradient problem.
\end{problembox}

\begin{problembox}[title= Forward Propagation Calculation]

Given a simple network:
- Input: $\vx = [1, 2]\tp$
- Weight matrix: $\mW^{(1)} = \begin{bmatrix} 0.5 & 1.0 \\ -0.3 & 0.8 \end{bmatrix}$
- Bias: $\vb^{(1)} = [0.1, -0.2]\tp$
- Activation: ReLU

Calculate the output of the hidden layer step by step.
\end{problembox}

\begin{problembox}[title= Chain Rule Application]

For the composite function $f(x) = (x^2 + 1)^3$:
a) Compute $\frac{df}{dx}$ using the chain rule
b) If $x = 2$, what is the numerical value of the derivative?

Relate this to how gradients flow through neural network layers.
\end{problembox}

\begin{problembox}[title= Backpropagation Setup]

For a 2-layer network with MSE loss:
$$\mathcal{L} = \frac{1}{2}(y - \hat{y})^2$$

Where $\hat{y} = \sigma(\vw_2^T \sigma(\vw_1^T \vx + b_1) + b_2)$

Set up the backpropagation equations to compute $\frac{\partial \mathcal{L}}{\partial \vw_1}$.
\end{problembox}

\begin{problembox}[title= Gradient Descent Parameter Update]

Given current weights $\vw = [0.5, -0.2, 0.8]^T$, gradients $\nabla \mathcal{L} = [0.1, -0.05, 0.3]^T$, and learning rate $\alpha = 0.01$:

a) Calculate the updated weights after one gradient descent step
b) If the loss decreases, what does this indicate about the gradient direction?
\end{problembox}

\begin{problembox}[title= Network Capacity Analysis]

Compare two networks for a classification task with 1000 training examples:
- Network A: 10 inputs, 5 hidden neurons, 2 outputs
- Network B: 10 inputs, 50 hidden neurons, 2 outputs

a) Calculate the number of parameters for each network
b) Which network is more likely to overfit? Why?
c) How would you prevent overfitting in the larger network?
\end{problembox}

\begin{problembox}[title= Vanishing Gradient Problem]

Consider a 5-layer network where each layer multiplies the gradient by 0.1:

a) What is the gradient magnitude at the first layer relative to the output layer?
b) How does this affect learning in early layers?
c) Suggest two techniques to mitigate this problem.
\end{problembox}

\begin{problembox}[title= Multi-class Classification]

Design a neural network for classifying images into 10 categories:
- Input: 784 pixels (28×28 image)
- Hidden layers: Your choice
- Output: 10 classes

a) What activation function should you use in the output layer?
b) What loss function is appropriate?
c) Write the mathematical form of the output layer computation.
\end{problembox}

\begin{problembox}[title= Computational Graph]

For the expression $z = (x + y) \times (x - y)$ where $x = 3$ and $y = 2$:

a) Draw the computational graph
b) Compute the forward pass values at each node
c) Compute the backward pass (gradients) using backpropagation
d) Verify by computing $\frac{\partial z}{\partial x}$ and $\frac{\partial z}{\partial y}$ analytically
\end{problembox}

\begin{problembox}[title= Learning Rate Analysis]

You're training a neural network and observe the following loss behavior:
- Learning rate 0.001: Loss decreases slowly and steadily
- Learning rate 0.1: Loss oscillates but generally decreases
- Learning rate 1.0: Loss increases or diverges

a) Explain what's happening in each case
b) What learning rate would you choose and why?
c) Describe adaptive learning rate methods that could help.
\end{problembox}

\begin{problembox}[title= Universal Approximation]

The Universal Approximation Theorem states that neural networks can approximate any continuous function:

a) What are the minimum requirements for this theorem to hold?
b) Does this guarantee that gradient descent will find the approximating network?
c) What practical limitations exist despite this theoretical result?
\end{problembox}

\begin{problembox}[title= Regularization Techniques]

List and explain three regularization techniques for neural networks:

a) How does each technique prevent overfitting?
b) Which technique would you apply to each layer type (input, hidden, output)?
c) How do these techniques affect the training and inference phases differently?
\end{problembox}

\begin{problembox}[title= Initialization Strategies]

Proper weight initialization is crucial for neural network training:

a) Why is initializing all weights to zero problematic?
b) Explain Xavier/Glorot initialization and when to use it
c) For a ReLU network with layers of sizes [100, 50, 25, 10], what variance should you use for initializing the weights of the second layer?
\end{problembox}

\begin{problembox}[title= Real-world Application]

Design a complete neural network solution for predicting house prices:

Input features: [square_feet, bedrooms, bathrooms, age, location_score]

a) Propose a network architecture with justification
b) What preprocessing steps would you apply to the inputs?
c) What loss function and evaluation metrics would you use?
d) How would you handle categorical variables like neighborhood?
e) Describe your training strategy including validation and hyperparameter tuning.
\end{problembox}

\end{document}