\documentclass{article}
\usepackage[utf8]{inputenc}
\usepackage{amsmath, amssymb, amsfonts}
\usepackage{bm}
\usepackage{graphicx}
\usepackage{xcolor}
\usepackage{tcolorbox}
\usepackage[bookmarks=false]{hyperref}
\usepackage{url}
\usepackage{booktabs}
\usepackage{array}
\usepackage{multirow}
\usepackage{enumitem}
\usepackage{tikz}
\usetikzlibrary{arrows.meta, positioning, shapes.geometric}
\usepackage[margin=1in]{geometry}
\usepackage{fancyhdr}
\usepackage{titlesec}

\pagestyle{fancy}
\fancyhf{}
\rhead{Tutorial: Ensemble Methods}
\lhead{ES335 - Machine Learning}
\rfoot{Page \thepage}

\titleformat{\section}{\Large\bfseries\color{blue!75!black}}{\thesection}{1em}{}
\titleformat{\subsection}{\large\bfseries\color{blue!60!black}}{\thesubsection}{1em}{}

\usepackage{../../shared/styles/conventions}

\newcounter{exercise}
\setcounter{exercise}{0}

\title{\textbf{Tutorial: Ensemble Methods} \\ \textit{Cheat Sheet and Practice Problems}}
\author{ES335 - Machine Learning \\ IIT Gandhinagar}
\date{\today}

\begin{document}

\maketitle

\section{Summary from Slides}

\subsection{Basic Concept}

\textbf{Ensemble Learning}: Use multiple models for prediction
\begin{itemize}
    \item Most winning entries of Kaggle competitions use ensemble learning
    \item Combines predictions from multiple base learners
    \item Often outperforms individual models
\end{itemize}

\textbf{Classification Example}:
\begin{itemize}
    \item Classifier 1: Good
    \item Classifier 2: Good  
    \item Classifier 3: Bad
    \item Majority Voting: Predict "Good"
\end{itemize}

\textbf{Regression Example}:
\begin{itemize}
    \item Regressor 1: 20
    \item Regressor 2: 30
    \item Regressor 3: 30
    \item Averaging: Predict $\frac{80}{3} = 26.67$
\end{itemize}

\subsection{Intuition and Conditions}

\textbf{Why Ensembles Work}:
\begin{itemize}
    \item Individual models make different types of errors
    \item Combining reduces overall error through error averaging
    \item Diversity among base learners is crucial
\end{itemize}

\textbf{Necessary and Sufficient Conditions}:
\begin{itemize}
    \item Base learners should be better than random guessing
    \item Base learners should be diverse (make different errors)
    \item Independence of base learner errors is ideal
\end{itemize}

\subsection{Random Forests}

\textbf{Key Concepts}:
\begin{itemize}
    \item Ensemble of decision trees
    \item Uses bootstrap sampling (bagging)
    \item Random feature selection at each split
    \item Reduces overfitting compared to single deep trees
\end{itemize}

\textbf{Representation}:
\begin{itemize}
    \item Limited depth decision trees have limited representation
    \item Random Forests expand representation through combination
    \item Better generalization than individual trees
\end{itemize}

\subsection{Quantitative Perspective}

\textbf{Error Reduction}: If individual classifiers have error rate $\epsilon < 0.5$ and are independent, ensemble error decreases exponentially with number of classifiers

\textbf{Bias-Variance Trade-off}:
\begin{itemize}
    \item Bagging reduces variance
    \item Boosting can reduce both bias and variance
    \item Ensemble typically has lower generalization error
\end{itemize}

\section{Practice Problems}

\begin{tcolorbox}[colback=gray!5!white,colframe=gray!75!black,title=Problem \stepcounter{exercise}: Basic Ensemble Voting]

Three binary classifiers make the following predictions on 5 test samples:
\begin{center}
\begin{tabular}{|c|c|c|c|c|}
\hline
Sample & Classifier 1 & Classifier 2 & Classifier 3 & True Label \\
\hline
1 & 0 & 1 & 1 & 1 \\
2 & 1 & 1 & 0 & 1 \\
3 & 0 & 0 & 1 & 0 \\
4 & 1 & 0 & 0 & 0 \\
5 & 1 & 1 & 1 & 1 \\
\hline
\end{tabular}
\end{center}

a) Calculate individual classifier accuracies
b) Calculate ensemble accuracy using majority voting
c) Which performs better: individual classifiers or ensemble?
\end{tcolorbox}

\begin{tcolorbox}[colback=gray!5!white,colframe=gray!75!black,title=Problem \stepcounter{exercise}: Regression Ensemble]

Four regression models predict house prices (in \$1000s):
\begin{center}
\begin{tabular}{|c|c|c|c|c|c|}
\hline
House & Model 1 & Model 2 & Model 3 & Model 4 & True Price \\
\hline
A & 250 & 280 & 240 & 260 & 255 \\
B & 180 & 200 & 170 & 190 & 185 \\
C & 320 & 310 & 330 & 315 & 318 \\
\hline
\end{tabular}
\end{center}

a) Calculate individual model MSEs
b) Calculate ensemble MSE using simple averaging
c) Try weighted averaging with weights [0.4, 0.3, 0.2, 0.1] - does it improve?
\end{tcolorbox}

\begin{tcolorbox}[colback=gray!5!white,colframe=gray!75!black,title=Problem \stepcounter{exercise}: Diversity Analysis]

Two classifiers have the following error patterns on 100 samples:
- Classifier A: Correct on samples 1-70, wrong on 71-100
- Classifier B: Correct on samples 21-90, wrong on 1-20 and 91-100

a) What is the individual accuracy of each classifier?
b) On how many samples do both classifiers agree?
c) What is the ensemble accuracy using majority voting?
d) Calculate the diversity between the two classifiers
\end{tcolorbox}

\begin{tcolorbox}[colback=gray!5!white,colframe=gray!75!black,title=Problem \stepcounter{exercise}: Bootstrap Sampling]

You have a training set with 6 samples: [A, B, C, D, E, F]

a) Create 3 different bootstrap samples of size 6
b) Which samples might be out-of-bag for each bootstrap?
c) How does bootstrap sampling introduce diversity?
d) What fraction of original samples are expected to be in each bootstrap sample?
\end{tcolorbox}

\begin{tcolorbox}[colback=gray!5!white,colframe=gray!75!black,title=Problem \stepcounter{exercise}: Random Forest Construction]

Design a Random Forest with the following specifications:
- 5 trees
- Maximum depth of 3
- Consider $\sqrt{p}$ features at each split where $p=16$ total features

a) How many features are considered at each split?
b) Draw the structure for one possible tree
c) How do you combine predictions for classification vs regression?
d) What are the advantages over a single deep tree?
\end{tcolorbox}

\begin{tcolorbox}[colback=gray!5!white,colframe=gray!75!black,title=Problem \stepcounter{exercise}: Ensemble Error Analysis]

Given $n$ independent binary classifiers, each with error rate $\epsilon = 0.3$:

a) What is the ensemble error rate for $n=3$ using majority voting?
b) Calculate for $n=5$ and $n=7$
c) How does ensemble error change as $n$ increases?
d) What happens if $\epsilon > 0.5$?
\end{tcolorbox}

\begin{tcolorbox}[colback=gray!5!white,colframe=gray!75!black,title=Problem \stepcounter{exercise}: Feature Importance in Random Forests]

A Random Forest is trained on a dataset with features [Age, Income, Education, Location]:

After training, the feature importance scores are:
- Age: 0.4
- Income: 0.3  
- Education: 0.2
- Location: 0.1

a) Interpret these importance scores
b) How are feature importances calculated in Random Forests?
c) Which features would you select for a simpler model?
d) How does random feature selection affect importance calculation?
\end{tcolorbox}

\begin{tcolorbox}[colback=gray!5!white,colframe=gray!75!black,title=Problem \stepcounter{exercise}: Bias-Variance in Ensembles]

Compare bias and variance for:
- Single decision tree (depth 10)
- Random Forest (100 trees, depth 5)
- Bagged trees (100 trees, depth 10)

a) Which has highest bias? Why?
b) Which has highest variance? Why?
c) How does bootstrap sampling affect bias and variance?
d) What is the bias-variance trade-off in each method?
\end{tcolorbox}

\begin{tcolorbox}[colback=gray!5!white,colframe=gray!75!black,title=Problem \stepcounter{exercise}: Ensemble Combination Methods]

You have 5 binary classifiers with the following probability outputs for a sample:
[0.8, 0.3, 0.6, 0.9, 0.4]

Calculate the ensemble prediction using:
a) Simple majority voting (threshold 0.5)
b) Average probability  
c) Weighted average with weights [0.3, 0.1, 0.2, 0.3, 0.1]
d) Maximum probability
e) Which methods agree and which disagree?
\end{tcolorbox}

\begin{tcolorbox}[colback=gray!5!white,colframe=gray!75!black,title=Problem \stepcounter{exercise}: Out-of-Bag Error Estimation]

In a Random Forest with 100 trees:

a) What fraction of samples are out-of-bag for each tree on average?
b) How is out-of-bag error calculated?
c) Why is OOB error a good estimate of generalization error?
d) Compare OOB error with k-fold cross-validation - advantages and disadvantages?
\end{tcolorbox}

\begin{tcolorbox}[colback=gray!5!white,colframe=gray!75!black,title=Problem \stepcounter{exercise}: Ensemble Diversity Measures]

Three classifiers make predictions on 8 samples:
\begin{center}
\begin{tabular}{|c|c|c|c|}
\hline
Sample & C1 & C2 & C3 \\
\hline
1 & 1 & 1 & 1 \\
2 & 1 & 1 & 0 \\
3 & 1 & 0 & 1 \\
4 & 1 & 0 & 0 \\
5 & 0 & 1 & 1 \\
6 & 0 & 1 & 0 \\
7 & 0 & 0 & 1 \\
8 & 0 & 0 & 0 \\
\hline
\end{tabular}
\end{center}

a) Calculate pairwise disagreement between classifiers
b) Calculate Q-statistic for each pair
c) Which pair is most diverse?
d) How does diversity relate to ensemble performance?
\end{tcolorbox}

\begin{tcolorbox}[colback=gray!5!white,colframe=gray!75!black,title=Problem \stepcounter{exercise}: Stacking vs Voting]

Design a two-level ensemble:
- Level 1: Train 4 base models (SVM, Random Forest, Logistic Regression, KNN)
- Level 2: Meta-learner combines Level 1 predictions

a) What are the inputs to the meta-learner?
b) How do you prevent overfitting in stacking?
c) Compare stacking with simple voting
d) When might stacking perform worse than voting?
\end{tcolorbox}

\begin{tcolorbox}[colback=gray!5!white,colframe=gray!75!black,title=Problem \stepcounter{exercise}: Computational Complexity]

Compare computational costs:
- Training: Single tree vs Random Forest (100 trees)
- Prediction: Single tree vs Random Forest
- Memory: Single tree vs Random Forest

a) How does ensemble size affect training time?
b) Can ensemble prediction be parallelized?
c) What are the memory requirements for storing 100 trees?
d) When might single models be preferred despite lower accuracy?
\end{tcolorbox}

\begin{tcolorbox}[colback=gray!5!white,colframe=gray!75!black,title=Problem \stepcounter{exercise}: Real-world Ensemble Design]

You're building an ensemble for credit card fraud detection with:
- Highly imbalanced data (0.1\% fraud)
- Need real-time predictions (<100ms)
- Interpretability required for regulatory compliance

a) What base models would you choose and why?
b) How would you handle class imbalance in each base model?
c) What ensemble combination method is most appropriate?
d) How would you ensure interpretability?
e) What constraints does the timing requirement impose?
\end{tcolorbox}

\begin{tcolorbox}[colback=gray!5!white,colframe=gray!75!black,title=Problem \stepcounter{exercise}: Advanced Challenge]

Prove mathematically that for binary classification with $n$ independent classifiers (each with error rate $\epsilon < 0.5$), the ensemble error using majority voting is:

$$P(\text{ensemble error}) = \sum_{k=\lceil n/2 \rceil}^{n} \binom{n}{k} \epsilon^k (1-\epsilon)^{n-k}$$

a) Derive this formula step by step
b) Show that this error decreases as $n$ increases when $\epsilon < 0.5$
c) What happens when $\epsilon = 0.5$?
d) Extend to the case where classifiers have different error rates
\end{tcolorbox}

\end{document}