\documentclass{beamer}
\usepackage{../../shared/styles/custom}
\usepackage{../../shared/styles/conventions}

\usepackage{amsfonts}

%\beamerdefaultoverlayspecification{<+->}
% \newcommand{\data}{\mathcal{D}}
% \newcommand\Item[1][]{%
% 	\ifx\relax#1\relax  \item \else \item[#1] \fi
% 	\abovedisplayskip=0pt\abovedisplayshortskip=0pt~\vspace*{-\baselineskip}}




\title{Support Vector Machines}
\date{\today}
\author{Nipun Batra}
\institute{IIT Gandhinagar}
\begin{document}
	\maketitle
	
	\begin{frame}{Non-Linearly Separable Data}
	    Data not separable in $\mathbb{R}$ \\
	    \vspace{0.5cm}
	    Can we still use SVM? \\
	    \vspace{0.5cm}
	    Yes!\\
	    How? Project data to a higher dimensional space.
	\end{frame}
	\begin{frame}{Projection/Transformation Function}
	    \begin{equation*}
	        \phi : \mathbb{R}^{d} \rightarrow \mathbb{R}^{D}
	    \end{equation*}
	    where, $d$ = original dimension \\
	    \hspace{1cm} $D$ = new dimension \\
	    In our example:\\
	    \hspace{1cm} $d = 1; D = 2$ 
	\end{frame}
	\begin{frame}{}
	    Linear SVM:\\
	    \hspace{1cm} Maximize\\
	    \begin{equation*}
	        L(\alpha) = \sum_{i=1}^{N}\alpha_{i} - \frac{1}{2}\sum_{i=1}^{N}\sum_{j=1}^{N}\alpha_{i}\alpha_{j}y_{i}y_{j}\overline{x_{i}}.\overline{x_{j}}
	    \end{equation*}
	    \hspace{1cm} such that constraints are satisfied.\\
	   \hspace{5cm} $\downarrow$\\
	   \hspace{3.8cm} Transformation ($\phi$)\\
	   \hspace{5cm} $\downarrow$\\
	   \begin{equation*}
	       L(\alpha) = \sum_{i=1}^{N}\alpha_{i} - \frac{1}{2}\sum_{i=1}^{N}\sum_{j=1}^{N}\alpha_{i}\alpha_{j}y_{i}y_{j}\phi(\overline{x_{i}}).\phi(\overline{x_{j}})
	   \end{equation*}
	\end{frame}
	\begin{frame}{Steps}
	    \begin{enumerate}
	        \item Compute $\phi(x)$ for each point \\
	        \begin{equation*}
	            \phi: \mathbb{R}^{d} \rightarrow \mathbb{R}^{D}
	        \end{equation*}
	        \item Computer dot products over $\mathbb{R}^{d}$ space
	    \end{enumerate}
	    \hspace{0.1cm} Q. If $D >> d$ \\
	    \hspace{0.6cm} Both steps are expensive!
	\end{frame}
	\begin{frame}{Kernel Trick}
	    
	\end{frame}
	
\end{document}
